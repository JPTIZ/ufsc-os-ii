\documentclass{report}

% Language packages
\usepackage[utf8]{inputenc}
\usepackage[T1]{fontenc}
\usepackage[portuguese]{babel}

% Code related packages/commands
\usepackage{caption}
\usepackage[newfloat]{minted}

\usepackage{hyperref}


\newcommand{\todo}[1]{{\color{red} #1}}

\title{\todo{A gente descobre depois}}

\author{João Gabriel Trombeta\\
        João Paulo Taylor Ienczak Zanette\\
        Ranieri Schroeder Althoff}
\date{\today}

\begin{document}

\maketitle

\tableofcontents

\chapter{Máquinas Virtuais}

\section{Sobre Máquinas Virtuais}

De maneira sucinta, máquinas virtuais (VM --- \textit{Virtual Machines}) são
computadores sendo executados por outros computadores. Chama-se de
\textbf{Guest} a máquina virtual em si, e de \textbf{Host} o \textit{hardware}
que oferece recursos para executar a VM\@.

\section{Hypervisor}

Também chamado de Monitor de Máquina Virtual (VMM --- \textit{Virtual Machine
Monitor}), um \textbf{Hypervisor} é um componente (seja \textit{hardware},
\textit{software} ou \textit{firmware}) responsável por criar e executar uma
VM, sendo o \textit{Host} o computador em que o Hypervisor é executado.

\section{Falhas em Hypervisors}

\subsection{Xen Hypervisor}

Uma falha de segurança detectada com relação a Hypervisors foi explorada no Xen
Hypervisor (criado pelo Xen Project, composto por membros da The Linux
Foundation), em que é possível chamar uma função arbitrária alterando a tabela
de \textit{Hypercalls} (semelhante a uma \textit{vtable}). Uma
\textit{Hypercall} é \textit{software trap} do Hypervisor para executar
operações privilegiadas (como atualizar tabelas de página).

Para explorar a falha, primeiramente é necessário descobrir a localização da
tabela de \textit{Hypercalls}. Para isso, deve-se procurar pela assinatura da
página. Porém, como a página não possui um formato tão previsível, é difícil de
localizá-la (o que é feito pelo \textit{checksum} do conteúdo da página). Em
compensação, a tabela de argumentos dos \textit{Hypercalls} possui um formato
previsível, já que seu conteúdo --- que é o número de argumentos de cada
\textit{Hypercall} --- é fixo, e portanto seu \textit{checksum} também é
previsível. Além disso, a tabela de argumentos sempre se encontra na página
seguinte à tabela de \textit{Hypercalls}, e portanto, ao encontrar uma, se tem
a localização da outra. Aliado à possibilidade de leitura e escrita de código
arbitrário, feito através de falhas nas regras de verificação de segurança de
escrita em páginas do \textit{Hypervision}, é possível então efetuar escape de
máquina virtual (i.e.\ acessar recursos do \textit{Host} que não pertencem à
máquina virtual).

\subsection{VMware ESXi e KVM}

\chapter{AMD Memory Encryption}

\section{Security Memory Encryption}

\section{Secure Encrypted Virtualization}

\section{Aplicação do SME e SEV}

\section{}

\bibliographystyle{ieeetr}
\nocite{*}
\bibliography{references}

\end{document}
