\documentclass{report}

\newcommand{\todo}[1]{{\color{red} #1}}

\title{\todo{A gente descobre depois}}

\author{João Gabriel Trombeta\\
        João Paulo Taylor Ienczak Zanette\\
        Ranieri Schroeder Althoff}
\date{\today}

% Language packages
\usepackage[utf8]{inputenc}
\usepackage[T1]{fontenc}
\usepackage[portuguese]{babel}

% Code related packages/commands
\usepackage{caption}
\usepackage[newfloat]{minted}


\begin{document}

\maketitle

\tableofcontents

\chapter{Estrutura geral de IoT}

\section{Arquitetura e Componentes}

Em uma visão de alto-nível, um ambiente IoT pode ser arquiteturalmente separado
em dois ambientes: uma \textit{Cloud} e uma \textit{Fog}. O segundo possui foco
na redução da latência e descentralização, sendo composto pelos nodo-sensores
(que na prática são microcontroladores ligados a sensores, fonte de energia e
um rádio), que extraem os dados mas, por terem recursos limitados, não lidam
com eles diretamente.

O ambiente \textit{Cloud} viria para terceirizar recursos (processamento,
armazenamento, etc.), já que os nodos que compõem a \textit{Fog} não teria o
suficiente deles para lidar com determinadas operações. Como exemplo, a
\textit{Cloud} pode oferecer uma interface para as aplicações, que poderão ser
desenvolvidas de forma mais trivial (sem ter noção da complexidade dos nodos
sensores e outros elementos da \textit{Fog}),

\section{Funcionamento geral}

Para comunicação entre nodos, a tecnologia RFID (\textit{Radio Frequence
Identification}) pode ser aplicada diretamente em tal ambiente, auxiliando na
redução de latência e energia, em que objetos observáveis e de interesse podem
receber \textit{tag}s reconhecidas por identificadores únicos e assim leitores
podem fazer consultas por elementos filtrando os que possuirem uma determinada
\textit{tag}, aproveitando a eficiência da transmissão via rádio.

\chapter{Gateways IoT}

\section{Funcionamento detalhado}

\section{Implementações existentes}

\section{Problemas e falhas \todo{(ver se há conteúdo relevante sobre)}}

\bibliographystyle{ieeetr}
\nocite{*}
\bibliography{references}

\end{document}
