\documentclass{report}

% Language packages
\usepackage[utf8]{inputenc}
\usepackage[T1]{fontenc}
\usepackage[portuguese]{babel}

% Code related packages/commands
\usepackage{caption}
\usepackage[newfloat]{minted}

\usepackage{hyperref}


\newcommand{\todo}[1]{{\color{red} #1}}

\title{\todo{A gente descobre depois}}

\author{João Gabriel Trombeta\\
        João Paulo Taylor Ienczak Zanette\\
        Ranieri Schroeder Althoff}
\date{\today}

\begin{document}

\maketitle

\tableofcontents

\chapter{Máquinas Virtuais}

\section{Sobre Máquinas Virtuais}

De maneira sucinta, máquinas virtuais (VM --- \textit{Virtual Machines}) são
computadores sendo executados por outros computadores. Chama-se de
\textbf{Guest} a máquina virtual em si, e de \textbf{Host} o \textit{hardware}
que oferece recursos para executar a VM\@.

\section{Hypervisor}

Também chamado de Monitor de Máquina Virtual (VMM --- \textit{Virtual Machine
Monitor}), um \textbf{Hypervisor} é um componente (seja \textit{hardware},
\textit{software} ou \textit{firmware}) responsável por criar e executar uma
VM, sendo o \textit{Host} o computador em que o Hypervisor é executado.

\chapter{AMD Memory Encryption}

\section{Security Memory Encryption}

\section{Secure Encrypted Virtualization}

\section{Aplicação}

\bibliographystyle{ieeetr}
\nocite{*}
\bibliography{references}

\end{document}
